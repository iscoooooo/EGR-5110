\documentclass{article}

\usepackage{fancyhdr, lastpage}
\usepackage[inline]{enumitem}
\usepackage{listings}
\usepackage[scaled=0.95]{inconsolata}  % Use a monospaced font, like Inconsolata
\usepackage{wasysym}
\usepackage{booktabs}
\usepackage{booktabs, multicol, multirow, array, threeparttable}
\usepackage{siunitx}
\usepackage{xfrac}
\usepackage{extramarks}
\usepackage{amsmath, amsthm, amsfonts, mathtools, empheq}
\usepackage{caption}
\usepackage[table]{xcolor}
\usepackage{tikz}
\usepackage[most]{tcolorbox}
\usepackage{pagecolor} %% for dark background
\usepackage{hyperref}
\usepackage{refcount}
\usepackage{subfigure}

\topmargin=-0.45in
\evensidemargin=0in
\oddsidemargin=0in
\textwidth=6.5in
\textheight=9.0in
\headsep=0.25in

\linespread{1.1}

% Define a command to print last page number without hyperlink
\newcommand*{\lastpagewithoutlink}{%
    \getpagerefnumber{LastPage}%
}

\pagestyle{fancy}
\lhead{\hmwkAuthorName}
\chead{\hmwkTitle}
\rhead{\hmwkClass}
\lfoot{\lastxmark}
\cfoot{Page \thepage \ of \lastpagewithoutlink}


\renewcommand\headrulewidth{0.4pt}
\renewcommand\footrulewidth{0.4pt}

\setlength\parindent{0pt}

%
% Create Problem Sections
%

\newcommand{\enterProblemHeader}[1]{
    \nobreak\extramarks{}{Problem \arabic{#1} continued on next page\ldots}\nobreak{}
    \nobreak\extramarks{Problem \arabic{#1} (continued)}{Problem \arabic{#1} continued on next page\ldots}\nobreak{}
}

\newcommand{\exitProblemHeader}[1]{
    \nobreak\extramarks{Problem \arabic{#1} (continued)}{Problem \arabic{#1} continued on next page\ldots}\nobreak{}
    \stepcounter{#1}
    \nobreak\extramarks{Problem \arabic{#1}}{}\nobreak{}
}

\setcounter{secnumdepth}{0}
\newcounter{partCounter}
\newcounter{homeworkProblemCounter}
\setcounter{homeworkProblemCounter}{1}
\nobreak\extramarks{Problem \arabic{homeworkProblemCounter}}{}\nobreak{}

%
% Homework Problem Environment
%
% This environment takes an optional argument. When given, it will adjust the
% problem counter. This is useful for when the problems given for your
% assignment aren't sequential. See the last 3 problems of this template for an
% example.
%
\newenvironment{homeworkProblem}[1][-1]{
    \ifnum#1>0
        \setcounter{homeworkProblemCounter}{#1}
    \fi
    \section{Problem \arabic{homeworkProblemCounter}}
    \setcounter{partCounter}{1}
    \enterProblemHeader{homeworkProblemCounter}
}{
    \exitProblemHeader{homeworkProblemCounter}
}

%
% Homework Details
%   - Title
%   - Due date
%   - Class
%   - Section/Time
%   - Instructor
%   - Author
%

\newcommand{\hmwkTitle}{Homework\ \#4}
\newcommand{\hmwkDueDate}{April 20, 2024}
\newcommand{\hmwkClass}{EGR 5110}
\newcommand{\hmwkClassTime}{}
\newcommand{\hmwkClassInstructor}{Professor Nissenson}
\newcommand{\hmwkAuthorName}{\textbf{Francisco Sanudo}}

%
% Title Page
%

\title{
    \vspace{2in}
    \textmd{\textbf{\hmwkClass:\ \hmwkTitle}}\\
    \normalsize\vspace{0.1in}\small{Due\ on\ \hmwkDueDate\ at 11:59pm}\\
    \vspace{0.1in}\large{\textit{\hmwkClassInstructor\ \hmwkClassTime}}
    \vspace{3in}
}

\author{\hmwkAuthorName}
\date{}

\renewcommand{\part}[1]{\textbf{\large Part \Alph{partCounter}}\stepcounter{partCounter}\\}

%
% More settings
%

% Table Settings
% \setlength{\tabcolsep}{5pt} % Gap before text starts
\renewcommand{\arraystretch}{2.5} % Cell Height Scaling
\setlength{\arrayrulewidth}{0.5mm} % Table Border Thickness
\arrayrulecolor{blue} % Table Border Color
\newcolumntype{s}{>{\columncolor{black!10}} c}

% Link Settings
\hypersetup{
    colorlinks=false,       % false: boxed links; true: colored links
    linkcolor=red,          % color of internal links (change box color with linkbordercolor)
    citecolor=green,        % color of links to bibliography
    urlcolor=cyan,           % color of external links
    pdftitle={EGR 5110 HW3}
}

% Color defintions
\definecolor{magenta}{RGB}{255,0,255}
\definecolor{cyan}{RGB}{0,255,255}
\definecolor{white}{RGB}{255,255,255}
\definecolor{red}{RGB}{255,0,0}
\definecolor{green}{RGB}{0,255,0}
\definecolor{orange}{RGB}{255,165,0}
\definecolor{yellow}{RGB}{255,255,0}
\definecolor{blue}{RGB}{10,10,255}

% Text color shortcuts
\newcommand{\cw}{\color{white}}
\newcommand{\cm}{\color{magenta}}
\newcommand{\cc}{\color{cyan}}
\newcommand{\cred}{\color{red}}
\newcommand{\cb}{\color{blue}}
\newcommand{\cg}{\color{green}}
\newcommand{\cy}{\color{yellow}}
\newcommand{\co}{\color{orange}}

%
% Various Helper Commands
%


% For derivatives
\newcommand{\deriv}[2]{\frac{d#1}{d#2}}

% For partial derivatives
\newcommand{\pderiv}[2]{\displaystyle \frac{\partial #1}{\partial #2}}

% Redefine \dfrac if you want all fractions to be in display style automatically
\newcommand{\ddfrac}[2]{\frac{\displaystyle #1}{\displaystyle #2}}

% Alias for the Solution section header
\newcommand{\solution}{\textbf{\large Solution}}

%% Box settings
% \setlength{\fboxsep}{9pt} % Adjust the padding thickness here
\setlength{\fboxrule}{1pt} % Adjust the border thickness here

% The \dimexpr\linewidth-2\fboxsep-2\fboxrule\relax calculation ensures that the width of the minipage is reduced by twice the padding and twice the border width, as there is padding and border on both the left and right sides.
\newcommand{\boxsettings}{\dimexpr\linewidth-2\fboxsep-2\fboxrule\relax}

% Paragraph box shortcut for tables
\newcommand{\PB}[2]{\parbox{#1}{\centering #2}}

%% make subscripts smaller
% \begingroup\lccode`~=`_
% \lowercase{\endgroup\def~}#1{_{\scriptscriptstyle#1}}
% \AtBeginDocument{\mathcode`_="8000 \catcode`_=12 }

%% small negative sign
\newcommand{\dashexp}{\scalebox{0.35}[0.5]{$-$}}
\newcommand{\dash}{\scalebox{0.5}[1.0]{$-$}}


\begin{document}

\maketitle

\pagebreak

% Make table of contents
\tableofcontents

% Make list of figures and tables
\listoffigures
\listoftables

\pagebreak

\section{Background}
A long rectangular fin is attached to a heat source. The fin is much longer (into the page) than its other
dimensions, so heat flow is approximately two-dimensional. Its left side is subjected to a constant base
temperature of 100 ${}^{\circ}$C and the other three sides experience convection. The fin's initial temperature is\\
40 ${}^{\circ}$C and the free stream air temperature is 25 ${}^{\circ}$C. \\

Below is a cross sectional view of the fin:

\begin{figure}[h]
    \centering
    \includegraphics[width=0.3\textwidth]{fig/fin.png}
    \caption{Long Rectangular Fin Attached to Heat Source}
    \label{fig:fin}
\end{figure}

The time-dependent temperature distribution is governed by the 2D heat diffusion equation

\begin{equation}
    \pderiv{T}{t} = \alpha \left( \pderiv{^2 T}{x^2} + \pderiv{^2 T}{y^2}\right)
    \label{eq:2DHeat}
\end{equation}

where $T$ is temperature and $\alpha$ is the thermal diffusivity coefficient.\\

\textbf{Goal}: Solve Equation~\eqref{eq:2DHeat} from an initial time $t_0$ to a final time $t_f$ for the temperature distribution across the 2D rectangular fin in Figure~\ref{fig:fin} (as a function of time) using a finite-difference method.\\

The following figure shows the coordinate system and general discretization of the fin:

\begin{figure}[h]
    \centering
    \includegraphics[width=0.45\textwidth]{fig/finSetup.png}
    \caption{Coordinate system and discretization of a 2D thin rectangular fin}
    \label{fig:finSetup}
\end{figure}

In this setup, the origin is fixed to the bottom-left corner of the fin, $\Delta x$ \& $\Delta y$ represents the node spacings, $N_x$ \& $N_y$ represents the final nodes in the $x$ \& $y$ direction, $i$ \& $j$ are the node indices for the $x$ \& $y$ direction, respectively.

\pagebreak

\section{Deriving Node Equations}

In the class notes, we derived the following node equations:
\begin{align}
    \shortintertext{Interior Nodes:}
    T_{i,j}^{k+1}     & = \lambda\left(T_{i-1}^k + T_{i,j-1}^k + T_{i+1,j}^k + T_{i,j+1}^k\right) + (1-4\lambda)T_{i,j}^k                         \\
    \shortintertext{Left Boundary:}
    T_{1,j}^{k+1}     & = T_{i,j}^k = T_b \\
    \shortintertext{Right Boundary (excluding corner nodes):}
    T_{N_x,j}^{k+1}   & = \lambda\left(2T_{N_x-1,j}^k + T_{N_x,j+1}^k + T_{N_x,j-1}^k + 2BT_{\infty}\right) + (1-4\lambda - 2B\lambda)T_{N_x,j}^k \\
    \shortintertext{Top-right corner node:}
    T_{N_x,N_y}^{k+1} & = \lambda\left(2T_{N_x-1,N_y}^k + T_{N_x,N_y-1}^k + 2BT_{\infty}\right) + (1-4\lambda - 4B\lambda)T_{N_x,N_y}^k
\end{align}
where $B = \ddfrac{h \Delta x}{k}$, $\lambda = \ddfrac{\alpha \Delta t}{(\Delta x)^2}$, $\alpha = \ddfrac{k}{\rho c_p}$.\\

We must derive the remaining node equations for the top boundary, lower boundary, and the bottom-right corner:

\begin{figure}[h]
    \centering
    \subfigure[Top Nodes]{\includegraphics[width=0.24\textwidth]{fig/topNodes.png}} 
    \hspace{0.5cm}
    \subfigure[Bottom Nodes]{\includegraphics[width=0.24\textwidth]{fig/bottomNodes.png}} 
    \hspace{0.5cm}
    \subfigure[Corner Node]{\includegraphics[width=0.24\textwidth]{fig/cornerNode.png}}
    \caption{Visualization of node configurations and boundary conditions}
    \label{fig:nodes}
\end{figure}

The energy balance at all boundary nodes is captured by:
\[
\dot{E}_{in} - \dot{E}_{out} + \dot{E}_{generated} = \dot{E}_{stored}    
\]
Since the rate of energy flowing out of the control volume is zero ($\dot{E}_{out}$) and there is no energy generation within the control volume ($\dot{E}_{generated}$ = 0), then the equation above simplifies to:
\[
\dot{E}_{in} = \dot{E}_{stored}    
\]
This implies that:
\begin{equation}
    \sum{\dot{Q}_{cond}} + \sum{\dot{Q}_{conv}} = m c_p \pderiv{T}{t}
    \label{eq:energy}
\end{equation}  
This equation represents the balance of heat energy at the node, accounting for both conductive and convective heat transfer rates and the rate of change of stored thermal energy within the fin material. For our numerical simulations, this equation can be discretized further to solve for the temperature distribution over time within the rectangular fin using finite-difference approximations.

\subsection{Top Boundary Nodes}

Consider the nodes located along the top boundary of the rectangular fin ($i$, $N_y$), excluding the corner nodes. These nodes are subject to conduction and convection with the free stream air temperature $T_{\infty}$.\\

\begin{figure}[h]
    \centering
    \includegraphics[width=0.8\textwidth]{fig/energyTop.png}
    \caption{Energy Balance on Top Nodes with Convection Boundary Condition}
    \label{fig:energyTop}
\end{figure}

Using Fourier's Law of Conduction ($\dot{Q}_{cond} = kA\pderiv{T}{x}$) and Newton's Law of Cooling ($\dot{Q}_{conv} = hA\Delta T$), let's look at the flow rates coming into the control volume and their discretization:

\begin{itemize}
    \item $\dot{Q}_{cond_{1}} = kA\ddfrac{\Delta T}{\Delta y} = k(\Delta x \Delta z) \ddfrac{T_{i,1-1}^k - T_{i,1}^k}{\Delta y}$
    \item $\dot{Q}_{cond_{2}} = kA\ddfrac{\Delta T}{\Delta x} = k(\ddfrac{\Delta y}{2} \Delta z) \ddfrac{T_{i+1,N_y}^k - T_{i,1}^k}{\Delta x}$
    \item $\dot{Q}_{cond_{3}} = kA\ddfrac{\Delta T}{\Delta x} = k(\ddfrac{\Delta y}{2} \Delta z) \ddfrac{T_{i-1,N_y}^k - T_{i,1}^k}{\Delta x}$
    \item $\dot{Q}_{conv} = hA\Delta T = h(\Delta x \Delta z)\left(T_{\infty} - T_{i,1}^k\right)$
\end{itemize}

Subsituting these expressions in Equation~\eqref{eq:energy} leads to:

\begin{multline*}
    \rho\left(\Delta x \ddfrac{\Delta y}{2} \Delta z\right) c_p \ddfrac{T_{i,1}^{k+1} - T_{i,1}^k}{\Delta t} = k(\Delta x \Delta z) \ddfrac{T_{i,1-1}^k - T_{i,1}^k}{\Delta y} \\
    + k(\ddfrac{\Delta y}{2} \Delta z) \ddfrac{T_{i+1,N_y}^k - T_{i,1}^k}{\Delta x} \\
    + k(\ddfrac{\Delta y}{2} \Delta z) \ddfrac{T_{i-1,N_y}^k - T_{i,1}^k}{\Delta x}
    + h(\Delta x \Delta z)\left(T_{\infty} - T_{i,1}^k\right)
\end{multline*}

Assuming $\Delta x = \Delta y$, then this simplifies to 
\begin{center}
    \noindent \fcolorbox{red}{white}{
        $T_{i,1}^{k+1} = \lambda\left(2T_{i,1-1}^k + T_{i+1,N_y}^k + T_{i-1,N_y}^k + 2BT_{\infty}\right) + (1-4\lambda-2B\lambda)T_{i,1}^k$
        }        
\end{center}


\subsection{Lower Boundary Nodes}
Consider the nodes located along the top boundary of the rectangular fin ($i$, $1$), excluding the corner nodes. These nodes are subject to conduction and convection with the free stream air temperature $T_{\infty}$.

\begin{figure}[h]
    \centering
    \includegraphics[width=0.7\textwidth]{fig/energyBottom.png}
    \caption{Energy Balance on Bottom Nodes with Convection Boundary Condition}
    \label{fig:energyBottom}
\end{figure}

Using Fourier's Law of Conduction and Newton's Law of Cooling, the flow rates coming into the control volume are:

\begin{itemize}
    \item $\dot{Q}_{cond_{1}} = kA\ddfrac{\Delta T}{\Delta y} = k(\Delta x \Delta z) \ddfrac{T_{i,2}^k - T_{i,1}^k}{\Delta y}$
    \item $\dot{Q}_{cond_{2}} = kA\ddfrac{\Delta T}{\Delta x} = k(\ddfrac{\Delta y}{2} \Delta z) \ddfrac{T_{i-1,1}^k - T_{i,1}^k}{\Delta x}$
    \item $\dot{Q}_{cond_{3}} = kA\ddfrac{\Delta T}{\Delta x} = k(\ddfrac{\Delta y}{2} \Delta z) \ddfrac{T_{i+1,1}^k - T_{i,1}^k}{\Delta x}$
    \item $\dot{Q}_{conv} = hA\Delta T = h(\Delta x \Delta z)\left(T_{\infty} - T_{i,1}^k\right)$
\end{itemize}

Subsituting these expressions in Equation~\eqref{eq:energy} leads to:

\begin{multline*}
    \rho\left(\Delta x \ddfrac{\Delta y}{2} \Delta z\right) c_p \ddfrac{T_{i,N_y}^{k+1} - T_{i,N_y}^k}{\Delta t} = k(\Delta x \Delta z) \ddfrac{T_{i,2}^k - T_{i,1}^k}{\Delta y} \\
    + k(\ddfrac{\Delta y}{2} \Delta z) \ddfrac{T_{i-1,1}^k - T_{i,1}^k}{\Delta x} \\
    + k(\ddfrac{\Delta y}{2} \Delta z) \ddfrac{T_{i+1,1}^k - T_{i,1}^k}{\Delta x}
    + h(\Delta x \Delta z)\left(T_{\infty} - T_{i,1}^k\right)
\end{multline*}

Assuming $\Delta x = \Delta y$, then this simplifies to 
\begin{center}
    \noindent \fcolorbox{red}{white}{
        $T_{i,1}^{k+1} = \lambda\left(2T_{i,2}^k + T_{i+1,1}^k + T_{i-1,1}^k + 2BT_{\infty}\right) + (1-4\lambda-2B\lambda)T_{i,1}^k$
        }        
\end{center}

\pagebreak 

\subsection{Bottom-right Corner Node}
Now, consider the node located on the bottom-right corner of the rectangular fin ($N_x$, $1$). This is also subject subject to conduction and convection with the free stream air temperature $T_{\infty}$.

\begin{figure}[h]
    \centering
    \includegraphics[width=0.5\textwidth]{fig/energyCorner.png}
    \caption{Energy Balance on Bottom-right Corner Node with Convection Boundary Condition}
    \label{fig:energyCorner}
\end{figure}

Using Fourier's Law of Conduction and Newton's Law of Cooling, the flow rates coming into the control volume are:

\begin{itemize}
    \item $\dot{Q}_{cond_{1}} = kA\ddfrac{\Delta T}{\Delta x} = k(\ddfrac{\Delta y}{2} \Delta z) \ddfrac{T_{N_x-1,1}^k - T_{N_x,1}^k}{\Delta x}$
    \item $\dot{Q}_{cond_{2}} = kA\ddfrac{\Delta T}{\Delta y} = k(\ddfrac{\Delta x}{2} \Delta z) \ddfrac{T_{N_x,2}^k - T_{N_x,1}^k}{\Delta y}$
    \item $\dot{Q}_{conv_{1}} = hA\Delta T = h(\ddfrac{\Delta x}{2} \Delta z) \left(T_{\infty} - T_{N_x,1}^k\right)$
    \item $\dot{Q}_{conv_{2}} = hA\Delta T = h(\ddfrac{\Delta y}{2} \Delta z)\left(T_{\infty} - T_{N_x,1}^k\right)$
\end{itemize}

Subsituting these expressions in Equation~\eqref{eq:energy} leads to:

\begin{multline*}
    \rho\left(\ddfrac{\Delta x}{2} \ddfrac{\Delta y}{2} \Delta z\right) c_p \ddfrac{T_{N_x,1}^{k+1} - T_{N_x,1}^k}{\Delta t} = k(\ddfrac{\Delta y}{2} \Delta z) \ddfrac{T_{N_x-1,1}^k - T_{N_x,1}^k}{\Delta x} \\
    + k(\ddfrac{\Delta x}{2} \Delta z) \ddfrac{T_{N_x,2}^k - T_{N_x,1}^k}{\Delta y} \\
    + h(\ddfrac{\Delta x}{2} \Delta z) \left(T_{\infty} - T_{N_x,1}^k\right)
    + h(\ddfrac{\Delta y}{2} \Delta z)\left(T_{\infty} - T_{N_x,1}^k\right)
\end{multline*}

Assuming $\Delta x = \Delta y$, then this simplifies to 
\begin{center}
    \noindent \fcolorbox{red}{white}{
        $T_{N_x,1}^{k+1} = 2\lambda\left(T_{N_x-1,1}^k + T_{N_x,2}^k + 2BT_{\infty}\right) + (1-4\lambda-4B\lambda)T_{N_x,1}^k$
        }        
\end{center}

\pagebreak

\section{Scenarios}


Understanding heat transfer and thermal properties is essential in engineering applications. Thermal conductivity ($k_{cond}$), thermal diffusivity ($\alpha$), and convective heat transfer coefficient ($h$) are key parameters that influence heat flow within materials. \\

Thermal conductivity ($k_{cond}$) denotes a material's ability to conduct heat, indicating how efficiently heat moves through the material. It depends on the material type and structure. Thermal diffusivity ($\alpha$) represents how quickly a material responds to temperature changes and is defined as the ratio of thermal conductivity ($k_{cond}$) to the product of density ($\rho$) and specific heat capacity ($c_p$). Convective heat transfer coefficient ($h$) quantifies heat transfer effectiveness between a solid surface and a fluid (like air or water). This coefficient varies based on fluid properties, flow conditions, and surface characteristics. In the study of temperature distribution in a thin rectangular fin over time, variations in $k_{cond}$, $\alpha$, and $h$ play crucial roles in determining transient and steady-state heat transfer behavior.\\

Let's analyze five scenarios based on the values of thermal conductivity ($k_{\textrm{cond}}$), thermal diffusivity ($\alpha$), and convection coefficient ($h$) listed in the table below:

\begin{table}[h]
    \centering
    \caption{Five Scenarios Using an Explicit Finite-Difference Method}
    \small
    \begin{threeparttable}
        \begin{tabular}{|s|c|c|c|c|c|c|c|c|} \hline
            {\cellcolor{yellow!50} \bf Scenario} & {\PB{1cm}{$k_{\textrm{cond}}$                                                                 \\$\left(\frac{\textrm{W}}{\textrm{m} \, {}^{\circ}\textrm{C}}\right)$}} & {\PB{1.2cm}{$\alpha$\\$\left(\frac{\textrm{m}^2}{\textrm{s}}\right)$}} & {\PB{1.1cm}{$h$\\$\left(\frac{\textrm{W}}{\textrm{m}^2 \, {}^{\circ}\textrm{C}}\right)$}} & {\PB{0.8cm}{$t_{ss}$\\(min)}} & {\PB{1.3cm}{$T_{\textrm{avg}}$ tip\\1D eqn* (${}^{\circ}$C)}} &  {\PB{1.3cm}{$T_{\textrm{avg}}$ tip\\sim* (${}^{\circ}$C)}} & {\PB{1.3cm}{$\dot{Q}$\\1D eqn* (W)}} & {\PB{1.3cm}{$\dot{Q}$\\sim* (W)}}
            \\ \hline
            Pure Al, fan high                    & 240                           & \SI{97e-6}{}  & 100 & 0.93  & 93.94 & 94.16 & 133.31 & 126.32 \\ \hline
            Pure Al, fan low                     & 240                           & \SI{97e-6}{}  & 10  & 0.99  & 99.35 & 99.37 & 14.02  & 13.27  \\ \hline
            AISI 302                             & 15                            & \SI{4e-6}{}   & 100 & 11.23 & 52.54 & 53.57 & 78.49  & 74.32  \\ \hline
            Low $k$, high $\alpha$               & 3                             & \SI{100e-6}{} & 100 & 0.055 & 28.77 & 29.40 & 37.61  & 34.43  \\ \hline
            High $k$, low $\alpha$               & 100                           & \SI{3e-6}{}   & 100 & 27.51 & 86.72 & 87.18 & 124.08 & 117.68 \\  \hline
        \end{tabular}
        \label{tab:Scenarios}
        \begin{tablenotes}
            \item [*] The average tip temperature and heat rate are the values at the end of the simulation, which are well past the time when the contour lines stop moving.
        \end{tablenotes}
    \end{threeparttable}
\end{table}

We'll first look at case 1 (Pure Al, fan high), and then show how adjusting these parameters affects the temperature distribution, time to reach steady-state, and heat rate into the fin for scenarios 2-5.

\pagebreak

\subsection{Scenario 1: Pure Aluminum, Fan High}

\begin{figure}[h]
    \centering
    \includegraphics[width=1\textwidth]{fig/contour1.png}
    \caption{Steady-State Temperature Distrubution for Scenario 1}
    \label{fig: Plot1}
\end{figure}

Simulation Parameters:
\begin{itemize}
    \item d$t$ = 0.0012
    \item $N_t$ = 500,000
    \item $B$ = $\SI{4.167e-4}{}$
    \item $\lambda$ = 0.1164
\end{itemize}

\pagebreak

\subsection{Scenario 2: Pure Aluminum, Fan Low}

\begin{figure}[h]
    \centering
    \includegraphics[width=1\textwidth]{fig/contour2.png}
    \caption{Steady-State Temperature Distrubution for Scenario 2}
    \label{fig: Plot2}
\end{figure}

Simulation Parameters:
\begin{itemize}
    \item d$t$ = 0.00015
    \item $N_t$ = 200,000
    \item $B$ = $\SI{4.167e-4}{}$
    \item $\lambda$ = 0.1455
\end{itemize}

Compared to scenario 1, the convective heat transfer coefficient $(h)$ was reduced while holding thermal diffusivity ($\alpha$) and thermal conducitivity $(k_{cond})$ constant. A lower $h$ value means slower heat transfer and potentially higher temperatures near the surface due to reduced cooling efficiency.\\

Effect of Adjustments:
\begin{enumerate}
    \item \textbf{Temperature Distrubution}:
        \begin{itemize}
            \item Surface temperatures are higher, with a slightly steeper temperature gradient across the fin.
            \item Distribution became more non-uniform, with higher temperatures near the base of the fin.
            \item Heat dissipation became less efficient
        \end{itemize}
    \item \textbf{Time to Steady State ($t_{ss}$)}:
        \begin{itemize}
            \item Slightly longer time to reach steady-state temperature due to slower heat dissipation.
        \end{itemize}
    \item \textbf{Heat Rate ($\dot{Q}$)}:
        \begin{itemize}
            \item Reducing $h$ decreased the convective heat transfer rate, $\dot{Q}_{conv}$, which means that less heat was transferred from the fin's surface to the surrounding air per unit time.
            \item This resulted in slower cooling of the fin's surface, and thus a smaller $\dot{Q}$ for dissipating heat.
        \end{itemize}
\end{enumerate}

\pagebreak

\subsection{Scenario 3: Stainless Steel, AISI 302}

\begin{figure}[h]
    \centering
    \includegraphics[width=1\textwidth]{fig/contour3.png}
    \caption{Steady-State Temperature Distrubution for Scenario 3}
    \label{fig: Plot3}
\end{figure}

Simulation Parameters:
\begin{itemize}
    \item d$t$ = 0.0015
    \item $N_t$ = 100,000
    \item $B$ = $\SI{6.667e-3}{}$
    \item $\lambda$ = 0.06
\end{itemize}

Compared to scenario 1, the convective heat transfer coefficient ($h$) was held constant whiile thermal dif-
fusivity ($\alpha$) and thermal conducitivity ($k_{cond}$) constant. Decreasing $\alpha$ means the material has a lower ability to respond to changes in temperature, implying slower heat propagation. Meanwhile, a decrease in $k$ means the material is less efficient at conducting heat.  \\

Effect of Adjustments:
\begin{enumerate}
    \item \textbf{Temperature Distrubution}:
        \begin{itemize}
            \item Decreasing $\alpha$ and $k$ led to a more non-uniform temperature distribution along the fin.
            \item Steeper temperature gradients are observed, especially near the heat source and fin base.
        \end{itemize}
    \item \textbf{Time to Steady State ($t_{ss}$)}:
        \begin{itemize}
            \item Both decreasing $\alpha$ and $k$ increased the time required for the fin to reach steady state.
            \item Slower heat propagation and reduced heat conduction prolonged the transient period, as seen in the animation produced by the solver.
        \end{itemize}
    \item \textbf{Heat Rate ($\dot{Q}$)}:
        \begin{itemize}
            \item Decreasing $\alpha$ and $k$ reduced the efficiency of heat transfer into the fin.
            \item As a result, the heat rate into the fin decreased because of slower heat propagation and reduced heat conduction through the material.
        \end{itemize}
\end{enumerate}

\pagebreak

\subsection{Scenario 4: Low $k$, high $\alpha$}

\begin{figure}[h]
    \centering
    \includegraphics[width=1\textwidth]{fig/contour4.png}
    \caption{Steady-State Temperature Distrubution for Scenario 4}
    \label{fig: Plot4}
\end{figure}

Simulation Parameters:
\begin{itemize}
    \item d$t$ = 0.0015
    \item $N_t$ = 200,000
    \item $B$ = $\SI{0.0333}{}$
    \item $\lambda$ = 0.15
\end{itemize}

Compared to scenario 1, the convective heat transfer coefficient ($h$) and thermal diffusivity ($\alpha$) were essentially held constant while thermal conductivity ($k_{cond}$) was reduced. Because $k_{cond}$ represents the material's ability to conduct heat, decreasing this value implies reduced heat conduction capability through the material. \\

Effect of Adjustments:
\begin{enumerate}
    \item \textbf{Temperature Distrubution}:
        \begin{itemize}
            \item Decreasing $k_{cond}$ resulted in a more non-uniform temperature distribution along the fin's length.
            \item The fin's base and areas near the heat source experienced higher temperatures due to reduced heat dissipation.
            \item Temperature gradients across the fin became steeper, indicating slower heat propagation through the material.
        \end{itemize}
    \item \textbf{Time to Steady State ($t_{ss}$)}:
        \begin{itemize}
            \item A lower $k_{cond}$ would typically prolong the time required for the fin to reach steady state. 
            \item Due to a high $\alpha$, this facilitated rapid heat propagation.
            \item Slower heat conduction due to low $k_{cond}$ combined with a high $\alpha$ and convective effects allowed for thermal equilibrium to be achieved rather quickly.
        \end{itemize}
    \item \textbf{Heat Rate ($\dot{Q}$)}:
        \begin{itemize}
            \item Decreasing $k_{cond}$ reduced the efficiency of heat transfer into the fin, since heat transferred more slowly through the material.
        \end{itemize}
\end{enumerate}

\pagebreak

\subsection{Scenario 5: High $k$, low $\alpha$}

\begin{figure}[h]
    \centering
    \includegraphics[width=1\textwidth]{fig/contour5.png}
    \caption{Steady-State Temperature Distrubution for Scenario 5}s
    \label{fig: Plot5}
\end{figure}

Simulation Parameters:
\begin{itemize}
    \item d$t$ = 0.0120
    \item $N_t$ = 200,000
    \item $B$ = 0.036
    \item $\lambda$ = 0.01
\end{itemize}

Instead of comparing to scenario 1, let's compare to scenario 4. The convective heat transfer coefficient ($h$) was held constant, while the thermal conductivity ($k_{cond}$) was increased and the thermal diffusivity ($\alpha$) were reduced. \\

Effect of Adjustments:
\begin{enumerate}
    \item \textbf{Temperature Distrubution}:
        \begin{itemize}
            \item Decreasing $\alpha$ and increasing $k_{cond}$ led to a more uniform temperature distribution across the fin.
            \item While lower $\alpha$ slows down heat propagation and causes steeper temperature gradients, higher $k_{cond}$ counteracted this by enhancing heat conduction and promoting a more even temperature profile.
        \end{itemize}
    \item \textbf{Time to Steady State ($t_{ss}$)}:
        \begin{itemize}
            \item Increasing $k_{cond}$ reduced the time required for the fin to reach steady state by accelerating heat conduction through the material.
            \item Despite the slower heat propagation due to lower $\alpha$, the combined effect of increasing $k_{cond}$ resulted in a shorter transient period before achieving thermal equilibrium, as seen in the animation.
        \end{itemize}
    \item \textbf{Heat Rate ($\dot{Q}$)}:
        \begin{itemize}
            \item The heat rate into the fin is influenced by both $\alpha$ and $k_{cond}$.
            \item Decreasing $\alpha$ reduces the efficiency of heat transfer into the fin by slowing down heat propagation.
            \item However, increasing $k_{cond}$ enhanced heat conduction and overall heat transfer efficiency, mitigating the impact of lower $\alpha$ on the heat rate. This is seen with the significantly higher heat rate value, compared to those in scenario 4.
        \end{itemize}
\end{enumerate}

\end{document}
